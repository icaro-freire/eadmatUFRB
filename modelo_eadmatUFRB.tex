\documentclass{eadmatUFRB}

\printanswers % <--------- Mostra as Soluções.

%=======================================
% Informações do Título da Lista
%=======================================
\TituloDaLista{Modelo Genérico} %---------> Coloque aqui o título da sua lista
\Aluno{Fulano de Tal, Cicrano Beltrano} %-> Coloque aqui seu nome ou o nome do grupo (abreviar, se necessário)
\DataDia{00} %----------------------------> Dia de entrega da Atividade Avaliativa
\DataMes{00} %----------------------------> Mês de entrega da Atividade Avaliativa
\NumeroDaLista{X} %-----------------------> Número da Lisda de Atividade (colocar números em algarismos romanos)
%=======================================

% Comandos simplificados ======================================================

%    \intc   ----> integral em curva fechada no sentido anti-horário.
%    \vazio  ----> conjunto vazio
%    \dd     ----> letra <<d>>  no modo romano (para usar em ``dx'' na integral)
%    \sen    ----> seno
%    \tg     ----> tangente
%    \arctg  ----> arcotangente
%    \Ln     ----> logarítmo maiúsculo
%    \Arg    ----> argumento maiúsculo
%    \Cis    ----> ('C' maiúsculo) abreveação para cos(x) +isen(x)
%    \cis    ----> ('c' minúsculo) abreveação para cis(x) 
%==============================================================================

%%%%%%%%%%%%%%%%%%%%%%%%%%%%%%%%%%%%%%%%%%%%%%%%%%%%%%%%%%%%%%%%%%%%%%%%%%%%%%%
% Início do Documento
%%%%%%%%%%%%%%%%%%%%%%%%%%%%%%%%%%%%%%%%%%%%%%%%%%%%%%%%%%%%%%%%%%%%%%%%%%%%%%%
\begin{document}
%
\titulo %----------------> Comando para gerar o cabeçalho estilizado (com logo da UFRB). Não apagar esse comando!
%
\begin{questions} %------> Ambiente para Questões (INÍCIO)

% Questão 01 ------------------------------------------------------------------
\question[3] Considere o conjunto de pontos complexos
\[
		z(t) = (3 + 2 \cos{t}) + (-1 + 3 \sen{t})\,i,
\]
onde $ t \in \mathbb{R} $.
\begin{parts}
   \part[1] Mostre que $ \left| z^{n}(t) \right| = \left| z(t) \right|^{n} $,
	   para todo $ n \in \mathbb{N} $.
		 \begin{solution}
		   Usaremos indução sobre $n$.
			 É natural pensarmos: existe algum natural que essa igualdade seja 
			 verdadeira?
			 Bom\ldots não é difícil perceber que para $n=1$ ela é verdadeira.
			 Para $n=2$ também é verdadeira\ldots
			 Então, passamos a pensar: ``Talvez só existe algum subconjunto de 
			 $ \mathbb{N} $ onde ela vale, mas não deve valer para todos os naturais''.
	
	     Seja, então, $ \mathcal{X} \subset \mathbb{N} $ definido por
	     \[
			   \mathcal{X} = 
				 \left\{\,
				 n \in \mathbb{N};\;\; \left| z^{n}(t) \right| = \left| z(t) \right|^{n}
				 \,\right\}
			 \]
	     ou seja, um subconjunto de número naturais onde se verifica a igualdade.
	     \begin{enumerate}
		     \item[(i)] $ 1 \in \mathcal{X} $; pois 
				   $
					   \left| z^{1}(t) \right| = 
					   \left| z(t) \right| =
					   \left| z(t) \right|^{1}
					 $.
		     \item[(ii)] Além disso, se certo natural $ k \in \mathcal{X} $, então 
		     \[
				   \left| z^{k + 1}(t) \right| = 
					 \left| z^{k}(t) \cdot z(t) \right| =
					 \left| z^{k}(t) \right| \cdot \left| z(t) \right| = 
					 \left| z(t) \right|^{k} \cdot \left| z(t) \right| =
					 \left| z(t) \right|^{k + 1}
				 \]
		     ou seja, ${k+1}\in\mathcal{X}$
	      \end{enumerate}
	      Ora, se o natural $ 1 $ está em $ \mathcal{X} $ e se para cada natural 
				$ k $ em $ \mathcal{X} $ o seu sucessor também está\ldots pelo Axioma~4
				de Peano, tem-se $ \mathcal{X} = \mathbb{N} $. Isso significa que a 
				igualdade não apenas é válida para uma parte dos naturais, mas para todo
				natural $ n $.
	    \end{solution}
	  \part[0.5] Calcule o valor de $ \left| z^{4}(t) \right| $, quando 
		 $ t = 3 \pi / 2 $.
	   \begin{solution}
	     Note que 
	     \[
	       z(3 \pi / 2) = 
				 \left[ 3 + 2 \cos{(3 \pi / 2)} \right] + 
				 \left[-1 + 3 \sen{(3 \pi / 2)} \right]\,i =
				 (3 + 2 \cdot 0) + [-1 + 3 \cdot (-1)]\,i =
				 3-4\,i.
	     \]
	     Ora, pelo item~(a), podemos fazer:
	     \[
	       \left| z^{4}(3 \pi / 2) \right| = 
				 \left| z(3 \pi / 2) \right|^4 =
				 \left| 3 - 4\,i \right|^4 = 
				 \left( \sqrt{3^2 + (-4)^2} \right)^4 =
				 25^2 =
				 \Ovalbox{$625$}.
	      \]
      \end{solution}
	  \part[0.5] O que representa, geometricamente, esse conjunto de números 
		 complexos?
	   \begin{solution}
	     Seja $ z = x + y\,i $.
	     Fazendo $ x = 3 + 2 \cos{t} $ e $ y = -1 + 3 \sen{t} $, temos
	     \begin{align}
	       \cos{t} &= \frac{x - 3}{2} \label{cos}\\
	       \sen{t} &= \frac{y + 1}{3} \label{sen}
	     \end{align}
	     Elevado ao quadrado as equações \eqref{cos} e \eqref{sen}, e somando 
			 membro a membro, encontramos:
	     \[
			   \left( \frac{x - 3}{2} \right)^2 + \left (\frac{y + 1}{3} \right)^2 =
				 \cos^{2}{t} + \sen^{2}{t}
				 \Longrightarrow
				 \Ovalbox{$\displaystyle\frac{(x-3)^2}{2^2}+\frac{(y-(-1))^2}{3^2}=1$}
			 \]
       Ou seja, uma \emph{Elipse} de centro $ (3, -1) $ e de semieixos $ a = 2 $
			 e $b=3$.
	     Veja a Figura~\ref{fig:elipse}.
	
	     \begin{minipage}{\textwidth}
	       \centering
	       \includegraphics[width=5cm]{elipse}
	       \captionof{figure}{Conjunto dos complexos $z(t)$ visto geometricamente.}
	       \label{fig:elipse}
	      \end{minipage}
      \end{solution}
\end{parts}
%------------------------------------------------------------------------------

\end{questions} %--------> Ambiente para Questões (TÉRMINO)
\end{document}