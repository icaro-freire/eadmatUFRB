\begin{solution}
  Usaremos indução sobre $n$.
		É natural pensarmos: existe algum natural que essa igualdade seja 
		verdadeira?
		Bom\ldots não é difícil perceber que para $n=1$ ela é verdadeira.
		Para $n=2$ também é verdadeira\ldots
		Então, passamos a pensar: ``Talvez só existe algum subconjunto de 
		$ \mathbb{N} $ onde ela vale, mas não deve valer para todos os naturais''.
	 Seja, então, $ \mathcal{X} \subset \mathbb{N} $ definido por
	 \[
	  \mathcal{X} = 
		 \left\{\,
		 n \in \mathbb{N};\;\; \left| z^{n}(t) \right| = \left| z(t) \right|^{n}
		 \,\right\}
	 \]
	 ou seja, um subconjunto de número naturais onde se verifica a igualdade.
	 \begin{enumerate}
	  \item[(i)] $ 1 \in \mathcal{X} $; pois 
		  $
			  \left| z^{1}(t) \right| = 
			  \left| z(t) \right| =
			  \left| z(t) \right|^{1}
	 		$.
		 \item[(ii)] Além disso, se certo natural $ k \in \mathcal{X} $, então 
		  \[
			  \left| z^{k + 1}(t) \right| = 
		 	 \left| z^{k}(t) \cdot z(t) \right| =
				 \left| z^{k}(t) \right| \cdot \left| z(t) \right| = 
				 \left| z(t) \right|^{k} \cdot \left| z(t) \right| =
				 \left| z(t) \right|^{k + 1}
				\]
		  ou seja, ${k+1}\in\mathcal{X}$
	 \end{enumerate}
	 Ora, se o natural $ 1 $ está em $ \mathcal{X} $ e se para cada natural 
		$ k $ em $ \mathcal{X} $ o seu sucessor também está\ldots pelo Axioma~4
		de Peano, tem-se $ \mathcal{X} = \mathbb{N} $. Isso significa que a 
		igualdade não apenas é válida para uma parte dos naturais, mas para todo
		natural $ n $.
\end{solution}